\documentclass[12pt]{fiche-rd-info}
\usepackage[utf8]{inputenc}
\usepackage[T1]{fontenc}

\begin{document}

\authorA{Pierre-Yves}{Hervo}
\authorB{Paul-François}{Jeau}

\begin{fichesuivi}{11 novembre 2013}{17 novembre 2013}
	\tempstravailA{14}{30}	
	\tempstravailB{15}{35}

\paragraph{}
	\begin{travaileffectue}
		\begin{itemize}
			\item Recherche et lecture d'articles, thèses traitant des améliorations locales du visage : moyenne, réalisée à $100$ \% ;
			\item Mise à jour de quelques fiches de lecture (en ayant revu l’intérêt de certains articles qui seront utiles pour l’état de l’art): moyenne, réalisée à $100$ \% ;
			\item Passage en revu de tous les articles trouvés pour en retirer certains moins pertinents: longue, réalisée à $100$ \% ;
			\item Rédaction de la partie état de l’art : simple, réalisée à $20$ \% ;
\end{itemize}
	\end{travaileffectue}

\begin{travailnoneffectue}
		\begin{itemize}
			\item Pas d’avancé dans la recherche d'outils réalisant des améliorations automatiques de portrait : nous avons revu l’ensemble des documents stockés dans Mendeley et procédé à un tri de ces derniers;
		\end{itemize}
	\end{travailnoneffectue}
		
\paragraph{}
	\begin{planification}
		\begin{itemize}
			\item Continuer la rédaction de l’état de l’art et la terminer 
			\item Comparer les différents outils, et les propositions effectuées dans l’état de l’art
			\item Rencontre M. Perreira Da Silva la semaine prochaine		
		\end{itemize}
	\end{planification}
\end{fichesuivi}


\end{document}
