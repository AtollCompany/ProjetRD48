\documentclass[12pt]{fiche-rd-info}
\usepackage[utf8]{inputenc}
\usepackage[T1]{fontenc}

\begin{document}

\authorA{Pierre-Yves}{Hervo}
\authorB{Paul-François}{Jeau}

\begin{fichesuivi}{18 novembre 2013}{24 novembre 2013}
	\tempstravailA{12}{05}
	\tempstravailB{19}{30}
\paragraph{}
	\begin{travaileffectue}
		\begin{itemize}
			\item Rédaction de la partie résumé du rapport avec sa classification ACM : simple, réalisée à $100$ \% ;
			\item Mise à jour de la partie introductive du rapport : simple, réalisée à $100$\%;
			\item Rédaction de la partie état de l’art : simple, réalisée à $70$ \% ;
			\item Rencontre avec M. Perreira Da Silva le mardi 19/11 : simple, réalisée à $100$ \% ;
		\end{itemize}
	\end{travaileffectue}

\begin{travailnoneffectue}
		\begin{itemize}
			\item Pas d’avancée notable dans la recherche d'outils implémentant des améliorations automatiques de portrait : toutes les méthodes que nous avons étudiées dans l’état de l’art n’ont pas fait l’objet d’implémentation		\end{itemize}
	\end{travailnoneffectue}
		
\paragraph{}
	\begin{echange}
		\begin{itemize}
			\item M. Perreira Da Silva nous a envoyé un nouvel article traitant de l'évaluation de la qualité esthétique des photos portraits. Cet article, qui est cette fois dédié à la photographie de portrait, apporte un complément sur ce que nous avions et permet d'affirmer certains critères qui sont apparus dans plusieurs articles.
			\item Nous avons eu une réunion mardi avec M. Perreira Da Silva. Lors de cette réunion, nous avons abordé plusieurs points.
			\item Nous hésitions sur la mise en forme de la partie sur les critères esthétiques dans l'état de l'art, et grâce à la discussion, nous les présenteront successivement et listeront les méthodes de calculs s'il y en a.
			\item Concernant la table listant les éléments bibliographiques, certaines entrées étaient peu complètes. Certaines entrées sous Mendeley n'avaient pas le bon type de document et certaines métadonnées étaient mal extraites.
			 \item Lors de la soutenance intermédiaire, le but est donc de présenter les critères d'évaluations esthétiques(et outils si possible) que nous avons identifiés. Une seconde partie sera dédiée aux propositions d'améliorations de photographie de portrait existantes. Chacun de ces points devra faire l'heure d'un comparatif avant de faire ressortir les avantages et limites de chaque approche.
			\item Toujours pour l'état de l'Art, il vaut mieux ne pas mettre de côté les traitements modifiant la structure des images afin de pouvoir mettre en avant les limites de ces solutions dans le comparatif et expliquer notre proposition.
			\item Au fil de nos lectures nous avons noté que pour la partie évaluation de la qualité des images, certains classificateurs fonctionnant au moyen d'un apprentissage supervisé, utilisaient des bases d'images pour l'entrainement et les tests. La question était de savoir si de telles bases étaient utilisées à Polytech. M. Perreira Da Silva nous a conseillé de nous pencher sur un rapport de projet de recherche de l'an dernier pour ce point. En effet le sujet traitant de la Beauté des Visages utilisait plusieurs bases de visages. Il s'agit des bases Karolinska et FEI. Une partie à ce sujet dans la partie évaluation de l'esthétique pourrait être utile, elle préciserait des méthodes d'évaluation globale et subjective(comme par exemple la possibilité de demander l'avis d'observateurs). Nous pourrions tout aussi bien effectuer des tests sur le trombinoscope de l'intranet.
			\item Nous nous sommes aussi intéressés aux outils pour manipuler les images et plus particulièrement permettant la détection de visages (en vue de pouvoir les améliorer). Nous avons lu que Matlab avait en 2012 mis à jour sa boite à outils pour la Vision par Ordinateur et que ses fonctions pouvaient être intéressantes. Il y a aussi la librairie OpenCV qui couplée au C++ peut donner de bons résultats. Nonobstant, M. Perreira n'est pas sûr que nous disposions de la toolbox Matlab. Après vérification, nous n'avons en effet pas cette toolbox disponible. Cette partie concernant les outils pour la manipulation peuvent apparaître dans la partie proposition mais ne devraient pas avoir une proportion très importante. En plus de cela, si nous le souhaitons plus tard, M. Perreira Da Silva peut nous envoyer du code Matlab pour estimer le flou d'une image.
			\item Dans la partie proposition, deux propositions pourraient être le bienvenu. En effet, nous pourrions présenter une approche que l'on pourrait de plus performante mais que nous ne pourrions mettre en place en raison de logiciels ou d'outils non disponibles, et une approche plus réaliste axée sur les manques d'une (ou d'une combinaison) des techniques existantes. 					\item Maintenant concernant le rapport, la rédaction se termine à la partie Propositions. Il ne faut pas oublier la partie auto-contrôle qui est à compléter.
		\end{itemize}
	\end{echange}		
		
\paragraph{}
	\begin{planification}
		\begin{itemize}
			\item Envoyer à M. Perreira Da Silva, au plus tard jeudi, une version bien avancée du rapport avant d'avoir un retour avant le rendu du dimanche 1er décembre.(état de l'art terminé, propositions formulées)
			\item Le diaporama sera aussi à réaliser afin de pouvoir compléter la fiche d'auto-contrôle
			\item Le Gantt effectif de la phase I est aussi à fournir
		\end{itemize}
	\end{planification}
\end{fichesuivi}









\end{document}
