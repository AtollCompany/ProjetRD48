\documentclass[12pt]{fiche-rd-info}
\usepackage[utf8]{inputenc}
\usepackage[T1]{fontenc}

\begin{document}

\authorA{Pierre-Yves}{Hervo}
\authorB{Paul-François}{Jeau}

\begin{fichesuivi}{04 novembre 2013}{10 novembre 2013}
	\tempstravailA{09}{45}
	\tempstravailB{09}{45}

\paragraph{}
	\begin{travaileffectue}
		\begin{itemize}
			\item Validation des fiches de lecture en attente : simple, réalisée à $100$ \% ;
			\item Lecture des derniers articles plus génériques de la première phase : simple, réalisée à $100$ \% ;
			\item Recherche et lecture d'articles, thèses traitant des améliorations locales du visage : moyenne, réalisée à $80$ \% ;
			\item Recherche d'outils réalisant des améliorations automatiques de portrait (dont la librairie OpenCV C++ pour la manipulation d'image, Makeup.Pho.to outil en ligne, PinkMirror outil en ligne mais avec sélection de zones manuelle, de nombreux outils proposaient simplement une amélioration basée sur une hausse de la luminosité de la photo pour gommer/masquer les défauts, ...) : moyenne, réalisée à $70$ \% ;
		\end{itemize}
	\end{travaileffectue}

\paragraph{}
	\begin{echange}
		\begin{itemize}
			\item M. Perreira Da Silva nous a envoyé un article auquel nous n'avions pas accès, à l'issue de notre mail de la semaine dernière.
		\end{itemize}
	\end{echange}	
	
\paragraph{}
	\begin{planification}
		\begin{itemize}
			\item Rattraper le temps manquant sur le projet (en raisons des jeux d'entreprise et du DS de Données Multimédia).
			\item Rédiger la partie bibliographique et mettre l'accent sur les techniques existantes permettant l'amélioration de photo de portrait (automatique).
			\item Comparer les différents outils
			\item Contacter M. Perreira Da Silva pour prendre un rendez-vous dans le courant de la semaine.
		\end{itemize}
	\end{planification}
\end{fichesuivi}

\end{document}
